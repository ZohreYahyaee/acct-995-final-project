% Options for packages loaded elsewhere
\PassOptionsToPackage{unicode}{hyperref}
\PassOptionsToPackage{hyphens}{url}
\PassOptionsToPackage{dvipsnames,svgnames,x11names}{xcolor}
%
\documentclass[
  letterpaper,
  DIV=11,
  numbers=noendperiod]{scrartcl}

\usepackage{amsmath,amssymb}
\usepackage{iftex}
\ifPDFTeX
  \usepackage[T1]{fontenc}
  \usepackage[utf8]{inputenc}
  \usepackage{textcomp} % provide euro and other symbols
\else % if luatex or xetex
  \usepackage{unicode-math}
  \defaultfontfeatures{Scale=MatchLowercase}
  \defaultfontfeatures[\rmfamily]{Ligatures=TeX,Scale=1}
\fi
\usepackage{lmodern}
\ifPDFTeX\else  
    % xetex/luatex font selection
\fi
% Use upquote if available, for straight quotes in verbatim environments
\IfFileExists{upquote.sty}{\usepackage{upquote}}{}
\IfFileExists{microtype.sty}{% use microtype if available
  \usepackage[]{microtype}
  \UseMicrotypeSet[protrusion]{basicmath} % disable protrusion for tt fonts
}{}
\makeatletter
\@ifundefined{KOMAClassName}{% if non-KOMA class
  \IfFileExists{parskip.sty}{%
    \usepackage{parskip}
  }{% else
    \setlength{\parindent}{0pt}
    \setlength{\parskip}{6pt plus 2pt minus 1pt}}
}{% if KOMA class
  \KOMAoptions{parskip=half}}
\makeatother
\usepackage{xcolor}
\setlength{\emergencystretch}{3em} % prevent overfull lines
\setcounter{secnumdepth}{-\maxdimen} % remove section numbering
% Make \paragraph and \subparagraph free-standing
\makeatletter
\ifx\paragraph\undefined\else
  \let\oldparagraph\paragraph
  \renewcommand{\paragraph}{
    \@ifstar
      \xxxParagraphStar
      \xxxParagraphNoStar
  }
  \newcommand{\xxxParagraphStar}[1]{\oldparagraph*{#1}\mbox{}}
  \newcommand{\xxxParagraphNoStar}[1]{\oldparagraph{#1}\mbox{}}
\fi
\ifx\subparagraph\undefined\else
  \let\oldsubparagraph\subparagraph
  \renewcommand{\subparagraph}{
    \@ifstar
      \xxxSubParagraphStar
      \xxxSubParagraphNoStar
  }
  \newcommand{\xxxSubParagraphStar}[1]{\oldsubparagraph*{#1}\mbox{}}
  \newcommand{\xxxSubParagraphNoStar}[1]{\oldsubparagraph{#1}\mbox{}}
\fi
\makeatother

\usepackage{color}
\usepackage{fancyvrb}
\newcommand{\VerbBar}{|}
\newcommand{\VERB}{\Verb[commandchars=\\\{\}]}
\DefineVerbatimEnvironment{Highlighting}{Verbatim}{commandchars=\\\{\}}
% Add ',fontsize=\small' for more characters per line
\usepackage{framed}
\definecolor{shadecolor}{RGB}{241,243,245}
\newenvironment{Shaded}{\begin{snugshade}}{\end{snugshade}}
\newcommand{\AlertTok}[1]{\textcolor[rgb]{0.68,0.00,0.00}{#1}}
\newcommand{\AnnotationTok}[1]{\textcolor[rgb]{0.37,0.37,0.37}{#1}}
\newcommand{\AttributeTok}[1]{\textcolor[rgb]{0.40,0.45,0.13}{#1}}
\newcommand{\BaseNTok}[1]{\textcolor[rgb]{0.68,0.00,0.00}{#1}}
\newcommand{\BuiltInTok}[1]{\textcolor[rgb]{0.00,0.23,0.31}{#1}}
\newcommand{\CharTok}[1]{\textcolor[rgb]{0.13,0.47,0.30}{#1}}
\newcommand{\CommentTok}[1]{\textcolor[rgb]{0.37,0.37,0.37}{#1}}
\newcommand{\CommentVarTok}[1]{\textcolor[rgb]{0.37,0.37,0.37}{\textit{#1}}}
\newcommand{\ConstantTok}[1]{\textcolor[rgb]{0.56,0.35,0.01}{#1}}
\newcommand{\ControlFlowTok}[1]{\textcolor[rgb]{0.00,0.23,0.31}{\textbf{#1}}}
\newcommand{\DataTypeTok}[1]{\textcolor[rgb]{0.68,0.00,0.00}{#1}}
\newcommand{\DecValTok}[1]{\textcolor[rgb]{0.68,0.00,0.00}{#1}}
\newcommand{\DocumentationTok}[1]{\textcolor[rgb]{0.37,0.37,0.37}{\textit{#1}}}
\newcommand{\ErrorTok}[1]{\textcolor[rgb]{0.68,0.00,0.00}{#1}}
\newcommand{\ExtensionTok}[1]{\textcolor[rgb]{0.00,0.23,0.31}{#1}}
\newcommand{\FloatTok}[1]{\textcolor[rgb]{0.68,0.00,0.00}{#1}}
\newcommand{\FunctionTok}[1]{\textcolor[rgb]{0.28,0.35,0.67}{#1}}
\newcommand{\ImportTok}[1]{\textcolor[rgb]{0.00,0.46,0.62}{#1}}
\newcommand{\InformationTok}[1]{\textcolor[rgb]{0.37,0.37,0.37}{#1}}
\newcommand{\KeywordTok}[1]{\textcolor[rgb]{0.00,0.23,0.31}{\textbf{#1}}}
\newcommand{\NormalTok}[1]{\textcolor[rgb]{0.00,0.23,0.31}{#1}}
\newcommand{\OperatorTok}[1]{\textcolor[rgb]{0.37,0.37,0.37}{#1}}
\newcommand{\OtherTok}[1]{\textcolor[rgb]{0.00,0.23,0.31}{#1}}
\newcommand{\PreprocessorTok}[1]{\textcolor[rgb]{0.68,0.00,0.00}{#1}}
\newcommand{\RegionMarkerTok}[1]{\textcolor[rgb]{0.00,0.23,0.31}{#1}}
\newcommand{\SpecialCharTok}[1]{\textcolor[rgb]{0.37,0.37,0.37}{#1}}
\newcommand{\SpecialStringTok}[1]{\textcolor[rgb]{0.13,0.47,0.30}{#1}}
\newcommand{\StringTok}[1]{\textcolor[rgb]{0.13,0.47,0.30}{#1}}
\newcommand{\VariableTok}[1]{\textcolor[rgb]{0.07,0.07,0.07}{#1}}
\newcommand{\VerbatimStringTok}[1]{\textcolor[rgb]{0.13,0.47,0.30}{#1}}
\newcommand{\WarningTok}[1]{\textcolor[rgb]{0.37,0.37,0.37}{\textit{#1}}}

\providecommand{\tightlist}{%
  \setlength{\itemsep}{0pt}\setlength{\parskip}{0pt}}\usepackage{longtable,booktabs,array}
\usepackage{calc} % for calculating minipage widths
% Correct order of tables after \paragraph or \subparagraph
\usepackage{etoolbox}
\makeatletter
\patchcmd\longtable{\par}{\if@noskipsec\mbox{}\fi\par}{}{}
\makeatother
% Allow footnotes in longtable head/foot
\IfFileExists{footnotehyper.sty}{\usepackage{footnotehyper}}{\usepackage{footnote}}
\makesavenoteenv{longtable}
\usepackage{graphicx}
\makeatletter
\def\maxwidth{\ifdim\Gin@nat@width>\linewidth\linewidth\else\Gin@nat@width\fi}
\def\maxheight{\ifdim\Gin@nat@height>\textheight\textheight\else\Gin@nat@height\fi}
\makeatother
% Scale images if necessary, so that they will not overflow the page
% margins by default, and it is still possible to overwrite the defaults
% using explicit options in \includegraphics[width, height, ...]{}
\setkeys{Gin}{width=\maxwidth,height=\maxheight,keepaspectratio}
% Set default figure placement to htbp
\makeatletter
\def\fps@figure{htbp}
\makeatother
% definitions for citeproc citations
\NewDocumentCommand\citeproctext{}{}
\NewDocumentCommand\citeproc{mm}{%
  \begingroup\def\citeproctext{#2}\cite{#1}\endgroup}
\makeatletter
 % allow citations to break across lines
 \let\@cite@ofmt\@firstofone
 % avoid brackets around text for \cite:
 \def\@biblabel#1{}
 \def\@cite#1#2{{#1\if@tempswa , #2\fi}}
\makeatother
\newlength{\cslhangindent}
\setlength{\cslhangindent}{1.5em}
\newlength{\csllabelwidth}
\setlength{\csllabelwidth}{3em}
\newenvironment{CSLReferences}[2] % #1 hanging-indent, #2 entry-spacing
 {\begin{list}{}{%
  \setlength{\itemindent}{0pt}
  \setlength{\leftmargin}{0pt}
  \setlength{\parsep}{0pt}
  % turn on hanging indent if param 1 is 1
  \ifodd #1
   \setlength{\leftmargin}{\cslhangindent}
   \setlength{\itemindent}{-1\cslhangindent}
  \fi
  % set entry spacing
  \setlength{\itemsep}{#2\baselineskip}}}
 {\end{list}}
\usepackage{calc}
\newcommand{\CSLBlock}[1]{\hfill\break\parbox[t]{\linewidth}{\strut\ignorespaces#1\strut}}
\newcommand{\CSLLeftMargin}[1]{\parbox[t]{\csllabelwidth}{\strut#1\strut}}
\newcommand{\CSLRightInline}[1]{\parbox[t]{\linewidth - \csllabelwidth}{\strut#1\strut}}
\newcommand{\CSLIndent}[1]{\hspace{\cslhangindent}#1}

\usepackage{booktabs}
\usepackage{longtable}
\usepackage{array}
\usepackage{multirow}
\usepackage{wrapfig}
\usepackage{float}
\usepackage{colortbl}
\usepackage{pdflscape}
\usepackage{tabu}
\usepackage{threeparttable}
\usepackage{threeparttablex}
\usepackage[normalem]{ulem}
\usepackage{makecell}
\usepackage{xcolor}
\KOMAoption{captions}{tableheading}
\makeatletter
\@ifpackageloaded{caption}{}{\usepackage{caption}}
\AtBeginDocument{%
\ifdefined\contentsname
  \renewcommand*\contentsname{Table of contents}
\else
  \newcommand\contentsname{Table of contents}
\fi
\ifdefined\listfigurename
  \renewcommand*\listfigurename{List of Figures}
\else
  \newcommand\listfigurename{List of Figures}
\fi
\ifdefined\listtablename
  \renewcommand*\listtablename{List of Tables}
\else
  \newcommand\listtablename{List of Tables}
\fi
\ifdefined\figurename
  \renewcommand*\figurename{Figure}
\else
  \newcommand\figurename{Figure}
\fi
\ifdefined\tablename
  \renewcommand*\tablename{Table}
\else
  \newcommand\tablename{Table}
\fi
}
\@ifpackageloaded{float}{}{\usepackage{float}}
\floatstyle{ruled}
\@ifundefined{c@chapter}{\newfloat{codelisting}{h}{lop}}{\newfloat{codelisting}{h}{lop}[chapter]}
\floatname{codelisting}{Listing}
\newcommand*\listoflistings{\listof{codelisting}{List of Listings}}
\makeatother
\makeatletter
\makeatother
\makeatletter
\@ifpackageloaded{caption}{}{\usepackage{caption}}
\@ifpackageloaded{subcaption}{}{\usepackage{subcaption}}
\makeatother

\ifLuaTeX
  \usepackage{selnolig}  % disable illegal ligatures
\fi
\usepackage{bookmark}

\IfFileExists{xurl.sty}{\usepackage{xurl}}{} % add URL line breaks if available
\urlstyle{same} % disable monospaced font for URLs
\hypersetup{
  pdftitle={Spillover Effects of Platform Entry: Predicting Third-Party Seller Exit on JD.com},
  pdfauthor={Zohre Yahyaee},
  colorlinks=true,
  linkcolor={blue},
  filecolor={Maroon},
  citecolor={Blue},
  urlcolor={Blue},
  pdfcreator={LaTeX via pandoc}}


\title{Spillover Effects of Platform Entry: Predicting Third-Party
Seller Exit on JD.com}
\author{Zohre Yahyaee}
\date{2025-05-17}

\begin{document}
\maketitle
\begin{abstract}
This study investigates whether first-party (1P) entry into product
spaces on JD.com increases the likelihood of third-party (3P) SKU exit.
Using SKU-level panel data and a range of classification models, we find
that exit appears to occur largely at random, which limits the
predictive power of all tested models. Nevertheless, we observe that
features capturing lack of recent orders and user interaction provide
some explanatory value. Our best-performing model, an XGBoost classifier
tuned via Bayesian optimization and interpreted using SHAP values,
identifies disengagement and product space crowding as important---but
ultimately insufficient---signals for accurately predicting exit
behavior.
\end{abstract}


\section{Introduction}\label{introduction}

\subsection{1.1 What is the research
question?}\label{what-is-the-research-question}

In this research, we investigate the dynamics of hybrid e-commerce
platforms, where the platform plays a dual role: acting both as a
retailer by selling its own products and as a platform operator
facilitating third-party sellers. This dual role can create a conflict
of interest, as the platform both supports and competes with third-party
sellers. Prior research suggests that the entry of the platform as a
first-party seller can negatively impact third-party sellers,
potentially leading some of them to exit the market Zhu \& Liu (2018);
Jiang et al. (2011). Our goal is to empirically examine this phenomenon
and understand how the platform's dual role influences third-party
seller behavior.

\subsection{1.2. Why is this research question
important?}\label{why-is-this-research-question-important}

This question is important because the hybrid platform business model
has become ubiquitous among online platforms globally. For example,
Amazon in the U.S. has utilized this model since the early 2000s, while
Alibaba and JD.com in China also follow similar practices Hu et al.
(2024); (\textbf{wells2018amazon?}). Understanding the impact of this
model is crucial for both policymakers and businesses, as it shapes
market competition and affects seller sustainability. While hybrid
platforms offer benefits through increased consumer reach and
operational efficiency, they also raise concerns about fair competition
and platform bias---particularly when the platform uses its data or
visibility advantages to undermine independent sellers Crawford et al.
(2022); (\textbf{he2020impact?}).

\footnote{I thank Professor Eric Weisbrod for guidance and patience for
  this challenging project. This project was completed for ACCT 995 at
  the University of Kansas.}

\subsection{1.3. What is the variable of
interest?}\label{what-is-the-variable-of-interest}

The main variable of interest is a binary classification outcome
indicating whether a third-party SKU deactivated (i.e., exited) within
the observed time period. This allows us to analyze the likelihood and
predictors of seller exit from the platform.

\section{Literature Review}\label{literature-review}

The rise of hybrid platforms---where the platform competes directly with
the sellers it hosts---has raised concerns about fair competition and
self-preferencing behavior. Foundational theoretical and empirical work
by Jiang et al. (2011) and Zhu \& Liu (2018) suggests that platform
entry is often strategic, targeting successful product spaces based on
third-party seller performance. While this allows platforms to reduce
uncertainty, it also creates potential harm: platform entry is
associated with reduced sales, lower visibility, and even exit for
affected third-party sellers. These findings support the view that
platform entry can crowd out independent merchants, though the magnitude
and mechanisms of these effects may vary across product types.

On one hand, Zhu \& Liu (2018) show that Amazon's entry reduces
third-party sales and discourages continued participation. Similarly,
Song et al. (2020) document that sellers anticipate platform entry and
adjust their product variety accordingly. These studies highlight the
competitive threat that platform entry can pose.

On the other hand, Deng et al. (2023) and Chi et al. (2022) present
evidence for positive spillovers: when the platform enters a product
space, it may boost overall demand through marketing and consumer
awareness, indirectly benefiting existing third-party sellers. In some
cases, third-party sales even rise after platform entry.

The divergence in these findings points to the role of context---such as
product popularity, brand strength, and timing of entry. Some products
may benefit from visibility spillovers, while others suffer from direct
substitution. However, few studies systematically examine whether such
spillover effects translate into exit behavior. In this project, we
shift the focus from pricing and sales to the more fundamental question
of market survival. By predicting third-party exit, we offer new insight
into how platform dynamics shape seller persistence under competitive
pressure.

\#Dataset Overview \#\# 3.1. What is the dataset about?

This study draws on SKU-level transaction data from JD.com, a major
hybrid e-commerce platform in China. The dataset spans a single
month---March 2018---and provides detailed information about each SKU's
commercial activity and platform presence. For each SKU, the data
includes:

\begin{itemize}
\tightlist
\item
  Product characteristics (e.g., brand, functional attributes),
\item
  Activation and deactivation dates (capturing SKU entry and exit),
\item
  User interaction measures (e.g., daily clicks, views), and
\item
  Purchase activity (e.g., daily order counts).
\end{itemize}

An important feature of this dataset is that JD.com maintains separate
listings for each seller-SKU pair, allowing for precise tracking of SKU
engagement over time. This structure provides a clean empirical setting
to analyze third-party (3P) seller behavior---particularly decisions to
deactivate a SKU---amid potential competition from platform-backed
first-party (1P) sellers Shen et al. (2024).

\subsection{3.2. What are the dataset
challenges?}\label{what-are-the-dataset-challenges}

This dataset offers rich transactional data, but several limitations
complicate the analysis:

\begin{itemize}
\tightlist
\item
  \textbf{Short observation window}: The data covers a single month
  (March 2018), which restricts our ability to observe long-term exit
  patterns or delayed responses to platform entry.
\item
  \textbf{Incomplete attribute data}: Many SKUs lack complete
  information on core product characteristics, making it difficult to
  consistently group them into functionally comparable product spaces.
  As a result, we focus our analysis on the subset of SKUs with
  non-missing values for both functional attributes and brand
  identifiers.
\item
  \textbf{Defining true entry events}: JD.com's structure makes it
  challenging to isolate clean instances of first-party entry. We
  mitigate this by identifying product spaces that transitioned from
  exclusive third-party to joint presence with first-party SKUs during
  the sample period Shen et al. (2024).
\end{itemize}

\subsection{3.3. Exploratory Data Analysis
(EDA)}\label{exploratory-data-analysis-eda}

We begin with a descriptive overview of the dataset:

\begin{itemize}
\tightlist
\item
  The dataset includes 31,876 unique SKUs, each linked to a seller type,
  brand, and product attributes.
\item
  Over 20 million user click events and approximately 500,000 purchase
  transactions are recorded, capturing detailed consumer interactions
  Shen et al. (2024).
\item
  Despite the scale, only 9,159 SKUs received at least one order during
  the observation window, highlighting the skewed nature of demand
  (\textbf{ernst2006estimating?}).
\item
  Preliminary comparisons reveal sharp contrasts between seller types:\\
  First-party SKUs received an average of 223 orders, while third-party
  SKUs averaged only 9 orders.
\end{itemize}

These disparities underscore the need to examine whether platform entry
is associated with reduced visibility or demand for third-party
sellers---raising the question of whether first-party participation
displaces existing market participants Zhu \& Liu (2018); Deng et al.
(2023).

\begin{verbatim}

active exited 
 29578   1123 
\end{verbatim}

\subsection{3.4 Feature Engineering}\label{feature-engineering}

To support predictive modeling, we engineered a set of variables that
reflect three core aspects of SKU behavior: the level of competition in
the product space, product characteristics, and indicators of platform
performance. The following paragraphs describe the purpose and
construction of each group of features.

\begin{itemize}
\item
  The feature \texttt{max\_non\_interacting\_streak} was created to
  capture disengagement. This variable records the maximum number of
  consecutive days during which a SKU received neither a click nor an
  order. It serves as a proxy for inactivity and is intended to reflect
  declining customer interest or visibility over time.
\item
  To control for variation in scale across categories, we constructed
  quantile-based rankings for prices, orders, and clicks. These were
  computed separately at three levels: within product space, within
  brand, and across the entire catalog. The resulting variables include
  price quantile within product space
  (\texttt{price\_quantile\_product}), click quantile within brand
  (\texttt{click\_quantile\_brand}), and others. These features help
  capture a SKU's relative position in terms of price and demand.
\item
  Several features were included to measure competition within a SKU's
  environment. These include the total number of SKUs in the product
  space, the number of first-party and third-party SKUs within the same
  brand and product group, and whether a first-party SKU entered the
  space. Together, these variables are intended to capture the intensity
  of local competition and potential crowding effects.
\item
  Product-level attributes were retained as categorical variables. The
  fields \texttt{attribute1} and \texttt{attribute2} represent the
  functional classification of each SKU. To avoid dropping observations
  with incomplete data, missing values were explicitly labeled as
  ``Missing'' and treated as a separate category.
\end{itemize}

Table @ref(tab:variable-description-table) presents the key variables
used in the prediction model. These variables span three conceptual
groups: (1) competitive pressure, such as the number of rival SKUs in a
product space; (2) inherent product or brand characteristics, including
categorical attributes and brand identity; and (3) platform performance
indicators, such as click volume, order quantiles, and recent
interaction history.

\begin{Shaded}
\begin{Highlighting}[]
\FunctionTok{library}\NormalTok{(knitr)}
\FunctionTok{library}\NormalTok{(tibble)}
\FunctionTok{library}\NormalTok{(kableExtra)}

\NormalTok{var\_descriptions }\OtherTok{\textless{}{-}}\NormalTok{ tibble}\SpecialCharTok{::}\FunctionTok{tibble}\NormalTok{(}
  \AttributeTok{Variable =} \FunctionTok{c}\NormalTok{(}
    \StringTok{"enter\_typ\_one"}\NormalTok{,}
    \StringTok{"total\_sku\_type\_1\_in\_product\_space"}\NormalTok{,}
    \StringTok{"total\_sku\_type\_2\_in\_product\_space"}\NormalTok{,}
    \StringTok{"total\_sku\_in\_product\_space"}\NormalTok{,}
    \StringTok{"total\_sku\_type\_1\_in\_product\_spacebrand\_ID"}\NormalTok{,}
    \StringTok{"total\_sku\_type\_2\_in\_product\_spacebrand\_ID"}\NormalTok{,}
    \StringTok{"NO\_type\_2\_activated\_in\_product\_spacebrand\_ID"}\NormalTok{,}
    \StringTok{"NO\_type\_1\_activated\_in\_product\_spacebrand\_ID"}\NormalTok{,}
    \StringTok{"price\_quantile\_overall"}\NormalTok{,}
    \StringTok{"price\_quantile\_product"}\NormalTok{,}
    \StringTok{"price\_quantile\_brand"}\NormalTok{,}
    \StringTok{"order\_quantile\_product"}\NormalTok{,}
    \StringTok{"order\_quantile\_brand"}\NormalTok{,}
    \StringTok{"order\_quantile\_overall"}\NormalTok{,}
    \StringTok{"click\_quantile\_product"}\NormalTok{,}
    \StringTok{"click\_quantile\_brand"}\NormalTok{,}
    \StringTok{"click\_quantile\_overall"}\NormalTok{,}
    \StringTok{"max\_non\_interacting\_streak"}\NormalTok{,}
    \StringTok{"attribute1"}\NormalTok{,}
    \StringTok{"attribute2"}
    
\NormalTok{  ),}
  \AttributeTok{Description =} \FunctionTok{c}\NormalTok{(}
    \StringTok{"Indicates whether a 1P SKU entered the product space."}\NormalTok{,}
    \StringTok{"Number of 1P SKUs in the product space."}\NormalTok{,}
    \StringTok{"Number of 3P SKUs in the product space."}\NormalTok{,}
    \StringTok{"Total number of SKUs in the product space."}\NormalTok{,}
    \StringTok{"Number of 1P SKUs in product space within brand."}\NormalTok{,}
    \StringTok{"Number of 3P SKUs in product space within brand."}\NormalTok{,}
    \StringTok{"Number of newly activated 3P SKUs in brand{-}product space."}\NormalTok{,}
    \StringTok{"Number of newly activated 1P SKUs in brand{-}product space."}\NormalTok{,}
    \StringTok{"SKU\textquotesingle{}s price quantile across all SKUs."}\NormalTok{,}
    \StringTok{"SKU\textquotesingle{}s price quantile within product space."}\NormalTok{,}
    \StringTok{"SKU\textquotesingle{}s price quantile within brand."}\NormalTok{,}
    \StringTok{"SKU\textquotesingle{}s order quantile within product space."}\NormalTok{,}
    \StringTok{"SKU\textquotesingle{}s order quantile within brand."}\NormalTok{,}
    \StringTok{"SKU\textquotesingle{}s order quantile overall."}\NormalTok{,}
    \StringTok{"SKU\textquotesingle{}s click quantile within product space."}\NormalTok{,}
    \StringTok{"SKU\textquotesingle{}s click quantile within brand."}\NormalTok{,}
    \StringTok{"SKU\textquotesingle{}s click quantile overall."}\NormalTok{,}
    \StringTok{"Max consecutive days with no interaction."}\NormalTok{,}
    \StringTok{"Product\textquotesingle{}s first functional attribute."}\NormalTok{,}
    \StringTok{"Product\textquotesingle{}s second functional attribute."}
    
\NormalTok{  ),}
  \AttributeTok{Category =} \FunctionTok{c}\NormalTok{(}
    \FunctionTok{rep}\NormalTok{(}\StringTok{"Competition{-}related"}\NormalTok{, }\DecValTok{8}\NormalTok{),}
    \FunctionTok{rep}\NormalTok{(}\StringTok{"Platform performance"}\NormalTok{, }\DecValTok{10}\NormalTok{),}
    \FunctionTok{rep}\NormalTok{(}\StringTok{"Inherent characteristic"}\NormalTok{, }\DecValTok{2}\NormalTok{)}
\NormalTok{  )}
\NormalTok{)}

\FunctionTok{kable}\NormalTok{(var\_descriptions, }\AttributeTok{caption =} \StringTok{"Descriptions of predictor variables used in the machine learning model."}\NormalTok{) }\SpecialCharTok{|\textgreater{}}
  \FunctionTok{kable\_styling}\NormalTok{(}\AttributeTok{bootstrap\_options =} \FunctionTok{c}\NormalTok{(}\StringTok{"striped"}\NormalTok{, }\StringTok{"hover"}\NormalTok{, }\StringTok{"condensed"}\NormalTok{, }\StringTok{"responsive"}\NormalTok{)) }\SpecialCharTok{|\textgreater{}}
  \FunctionTok{column\_spec}\NormalTok{(}\DecValTok{1}\NormalTok{, }\AttributeTok{bold =} \ConstantTok{TRUE}\NormalTok{) }\SpecialCharTok{|\textgreater{}}
  \FunctionTok{collapse\_rows}\NormalTok{(}\AttributeTok{columns =} \DecValTok{3}\NormalTok{, }\AttributeTok{valign =} \StringTok{"top"}\NormalTok{)}
\end{Highlighting}
\end{Shaded}

\begin{longtable}[t]{>{}lll}
\caption{Descriptions of predictor variables used in the machine learning model.}\\
\toprule
Variable & Description & Category\\
\midrule
\textbf{enter\_typ\_one} & Indicates whether a 1P SKU entered the product space. & \\
\cmidrule{1-2}\nopagebreak
\textbf{total\_sku\_type\_1\_in\_product\_space} & Number of 1P SKUs in the product space. & \\
\cmidrule{1-2}\nopagebreak
\textbf{total\_sku\_type\_2\_in\_product\_space} & Number of 3P SKUs in the product space. & \\
\cmidrule{1-2}\nopagebreak
\textbf{total\_sku\_in\_product\_space} & Total number of SKUs in the product space. & \\
\cmidrule{1-2}\nopagebreak
\textbf{total\_sku\_type\_1\_in\_product\_spacebrand\_ID} & Number of 1P SKUs in product space within brand. & \\
\cmidrule{1-2}\nopagebreak
\textbf{total\_sku\_type\_2\_in\_product\_spacebrand\_ID} & Number of 3P SKUs in product space within brand. & \\
\cmidrule{1-2}\nopagebreak
\textbf{NO\_type\_2\_activated\_in\_product\_spacebrand\_ID} & Number of newly activated 3P SKUs in brand-product space. & \\
\cmidrule{1-2}\nopagebreak
\textbf{NO\_type\_1\_activated\_in\_product\_spacebrand\_ID} & Number of newly activated 1P SKUs in brand-product space. & \multirow[t]{-8}{*}{\raggedright\arraybackslash Competition-related}\\
\cmidrule{1-3}\pagebreak[0]
\textbf{price\_quantile\_overall} & SKU's price quantile across all SKUs. & \\
\cmidrule{1-2}\nopagebreak
\textbf{price\_quantile\_product} & SKU's price quantile within product space. & \\
\cmidrule{1-2}\nopagebreak
\textbf{price\_quantile\_brand} & SKU's price quantile within brand. & \\
\cmidrule{1-2}\nopagebreak
\textbf{order\_quantile\_product} & SKU's order quantile within product space. & \\
\cmidrule{1-2}\nopagebreak
\textbf{order\_quantile\_brand} & SKU's order quantile within brand. & \\
\cmidrule{1-2}\nopagebreak
\textbf{order\_quantile\_overall} & SKU's order quantile overall. & \\
\cmidrule{1-2}\nopagebreak
\textbf{click\_quantile\_product} & SKU's click quantile within product space. & \\
\cmidrule{1-2}\nopagebreak
\textbf{click\_quantile\_brand} & SKU's click quantile within brand. & \\
\cmidrule{1-2}\nopagebreak
\textbf{click\_quantile\_overall} & SKU's click quantile overall. & \\
\cmidrule{1-2}\nopagebreak
\textbf{max\_non\_interacting\_streak} & Max consecutive days with no interaction. & \multirow[t]{-10}{*}{\raggedright\arraybackslash Platform performance}\\
\cmidrule{1-3}\pagebreak[0]
\textbf{attribute1} & Product's first functional attribute. & \\
\cmidrule{1-2}\nopagebreak
\textbf{attribute2} & Product's second functional attribute. & \multirow[t]{-2}{*}{\raggedright\arraybackslash Inherent characteristic}\\
\bottomrule
\end{longtable}

\section{Model Development and
Evaluation}\label{model-development-and-evaluation}

\subsection{4.1 Preprocessing
Strategies}\label{preprocessing-strategies}

To address class imbalance and missing values in the dataset, we
constructed three modeling recipes. Each recipe includes dummy encoding
for categorical variables and normalization of predictors but differs in
how it handles missing data and class imbalance.

\begin{itemize}
\tightlist
\item
  The first recipe applies upsampling to the minority class (exited
  SKUs) after omitting observations with missing values. This approach
  balances the dataset while retaining only complete cases.
\item
  The second recipe uses SMOTE (Synthetic Minority Oversampling
  Technique) to generate synthetic examples for the minority class,
  again following the removal of missing values.
\item
  The third recipe applies SMOTE without removing missing values. This
  allows for the inclusion of more data points but increases the risk of
  introducing noise due to incomplete records.
\end{itemize}

These three recipes were designed to explore the sensitivity of model
performance to different data preparation strategies.

\subsection{4.2 Train-Test Split and
Cross-Validation}\label{train-test-split-and-cross-validation}

To evaluate model performance, we split the dataset into training and
testing sets using an 80-20 stratified split. Stratification was applied
on the \texttt{exited} variable to preserve the original class
proportions in both subsets. This ensures that the rare event of SKU
exit remains represented in both training and testing phases.

For model tuning, we employed five-fold cross-validation on the training
data, again using stratification to maintain class balance within folds.
This setup improves the stability of performance estimates and helps
mitigate overfitting.

\subsection{4.3 Classification Models}\label{classification-models}

We implemented and tuned four classification algorithms to predict SKU
exit:

\begin{itemize}
\tightlist
\item
  A logistic regression model using the \texttt{glm} engine
\item
  An elastic net model using \texttt{glmnet}, with the penalty and
  mixing parameters tuned
\item
  A random forest model using the \texttt{ranger} engine with 500 trees
\item
  An XGBoost model using the \texttt{xgboost} engine, with tuned
  hyperparameters and a \texttt{scale\_pos\_weight} parameter set to
  emphasize the rare exited class
\end{itemize}

Of these, only the XGBoost model incorporated class weighting in
addition to SMOTE or upsampling. This additional weighting adjustment
was intended to further address the severe class imbalance in the
dataset.

\subsection{4.4 Workflow Setup and
Tuning}\label{workflow-setup-and-tuning}

Each model-recipe pair was combined into a workflow using the
workflowsets package and tuned using a grid search over 20
hyperparameter combinations. Tuning was conducted using five-fold
cross-validation on the training set, with stratification on the exited
outcome to maintain class proportions within each fold.

The following metrics were used to evaluate model performance across
folds:

\begin{itemize}
\tightlist
\item
  Accuracy
\item
  Area under the ROC curve (AUC)
\item
  Brier score for calibration
\item
  Precision
\item
  Recall
\item
  F1-score
\end{itemize}

For models like elastic net and XGBoost, tuning focused on
regularization parameters (penalty, mixture) and tree-specific settings
(learn\_rate, tree\_depth, etc.). The tune package handled the grid
search and stored predictions and workflows for downstream use. Parallel
processing was used to improve efficiency.

While this study used a fixed grid search, the workflow is fully
compatible with more advanced methods such as Bayesian optimization
(e.g., via tune\_bayes() in the finetune package), which could be
explored in future work.

\begin{verbatim}
unique notes:
---------------------------------------------------------
glm.fit: fitted probabilities numerically 0 or 1 occurred
\end{verbatim}

\subsection{4.5 Model Comparison
Summary}\label{model-comparison-summary}

To evaluate model performance across different preprocessing strategies
and algorithms, we compared workflows built on three recipes and four
classifiers. Figure @ref(fig:model-performance-summary) summarizes the
best scores for each workflow based on key metrics, including accuracy,
precision, recall, F1-score, AUC, and Brier score.

\begin{itemize}
\tightlist
\item
  The XGBoost model using the SMOTE recipe without missing value removal
  (\texttt{smote\_no\_na\_recipe\_xgboost}) consistently performed best,
  particularly in terms of recall and F1-score.
\item
  Logistic regression models performed the worst, especially under the
  upsample recipe, highlighting their limited capacity for capturing
  complex patterns in exit behavior.
\item
  Elastic net and random forest models performed moderately well but
  fell short of XGBoost on recall and overall F1-score.
\item
  Recall scores remained low across all workflows, emphasizing the
  inherent difficulty of identifying rare exit cases.
\end{itemize}

Based on these results, we selected the SMOTE-based XGBoost model with
retained missing values as the final model for evaluation and
interpretation.

\includegraphics{05-ML_final_files/figure-pdf/Plot best model performance-1.pdf}

\subsection{4.6 Model Selection
Rationale}\label{model-selection-rationale}

While overall accuracy and AUC were relatively high across several
workflows, recall was notably low---reflecting the rarity and
unpredictability of exit events in the dataset. Because our research
goal is to understand and predict SKU exit, recall and F1-score were
treated as the most important criteria in selecting a final model.

The XGBoost model trained with the SMOTE recipe that retained missing
values provided the best tradeoff between recall and overall
performance. Although this approach carries the risk of introducing
noise from unclean data, it was more effective in identifying exited
SKUs than other configurations. For this reason, it was selected as the
final model for interpretation and out-of-sample testing.

\subsection{4.7 Final Model Training and
Prediction}\label{final-model-training-and-prediction}

The selected workflow---XGBoost trained using SMOTE without removing
missing values---was finalized using the best hyperparameters identified
during tuning. This final model was retrained on the full training
dataset and then applied to the test data.

Predicted outputs include both class labels and class probabilities,
which were used to generate evaluation metrics and visualizations,
including the confusion matrix and SHAP-based model interpretation.

\subsection{4.8 Evaluation of Final Model
Performance}\label{evaluation-of-final-model-performance}

\subsubsection{4.8.1 Confusion Matrix}\label{confusion-matrix}

Figure (\textbf{ref?})(fig-conf-matrix) displays the confusion matrix
generated from the final XGBoost model's predictions on the test set.
The counts are as follows:

\begin{itemize}
\tightlist
\item
  True positives (bottom-right): 28 exited SKUs correctly predicted as
  exited
\item
  True negatives (top-left): 5,911 active SKUs correctly predicted as
  active
\item
  False positives (top-right): 201 active SKUs incorrectly predicted as
  exited
\item
  False negatives (bottom-left): 1 exited SKU incorrectly predicted as
  active
\end{itemize}

Although the model identifies a portion of the exited SKUs, the number
of false positives (201) significantly exceeds the true positives (28).
This pattern indicates a tendency to over-predict exit events, which
reduces the trustworthiness of its output in practice.

The model is especially prone to false alarms, flagging many SKUs as
exited when they are actually active. While only one exited SKU was
missed, this still matters in a rare-event prediction context. Overall,
the imbalance between correctly and incorrectly predicted exits weakens
the model's practical utility.

\includegraphics{05-ML_final_files/figure-pdf/Confusion matrix-1.pdf}

\subsubsection{4.8.2 Classification
Metrics}\label{classification-metrics}

The metrics reported in Table (\textbf{ref?})(tab:eval-metrics-table)
provide a broader view of the model's predictive performance:

Accuracy measures overall correctness, reflecting how many SKUs were
predicted correctly regardless of class.

Precision indicates how many of the SKUs predicted as exited were
actually exited. Low precision here reflects the large number of false
positives.

Recall captures the model's ability to detect exited SKUs. This is
particularly important in rare event detection tasks.

F1-score summarizes the balance between precision and recall. In this
case, it reflects moderate performance.

ROC AUC quantifies the model's ability to distinguish exited from active
SKUs across a range of classification thresholds.

These metrics confirm that the model achieves a reasonable balance
between recall and overall accuracy, but precision remains low due to
frequent over-prediction of exit. This supports the view that, while the
model detects some meaningful patterns, its predictions may not yet be
reliable enough for automated business decisions regarding third-party
seller exit.

\begin{verbatim}
# A tibble: 5 x 3
  .metric   .estimator .estimate
  <chr>     <chr>          <dbl>
1 accuracy  binary         0.967
2 precision binary         0.967
3 recall    binary         1.00 
4 f_meas    binary         0.983
5 roc_auc   binary         0.162
\end{verbatim}

\section{Global Model Explanation}\label{global-model-explanation}

To better understand which features influenced the model's predictions,
we applied three complementary global interpretation methods: (1)
internal gain-based importance via the vip package, (2)
permutation-based dropout loss using DALEX, and (3) partial dependence
analysis for top predictors.

\subsection{5.1. Variable Importance via XGBoost
(VIP)}\label{variable-importance-via-xgboost-vip}

Figure (\textbf{ref?})(fig-vip-xgb) ranks the top 20 predictors based on
XGBoost's internal gain metric. Notably, features like
max\_non\_interacting\_streak and click\_quantile\_overall appear most
influential, suggesting that prolonged inactivity and lack of visibility
are strong indicators of SKU exit. This aligns with the intuition that
disengagement and poor platform performance precede deactivation
decisions.

\begin{figure}

\centering{

\includegraphics{05-ML_final_files/figure-pdf/fig-vip-xgb-1.pdf}

}

\caption{\label{fig-vip-xgb}}

\end{figure}%

\subsection{5.2. Permutation-Based Variable Importance (Custom 0/1
Loss)}\label{permutation-based-variable-importance-custom-01-loss}

Figure (\textbf{ref?})(fig-custom-permutation-vi) shows the variable
importance results under a permutation scheme using a binary 0/1 loss
function. Variables such as max\_non\_interacting\_streak,
total\_sku\_type\_2\_in\_product\_space, and click\_quantile\_overall
are among the most important predictors, confirming earlier findings
that disengagement and competitive intensity are key drivers of SKU
exit.

Notably, the variable enter\_typ\_one, which indicates whether a
first-party SKU entered the same product space, ranks near the bottom in
terms of importance. This suggests that, in the context of this
predictive model, first-party entry was not a strong direct predictor of
third-party SKU deactivation. While entry may still play an indirect or
structural role, it does not appear to influence exit decisions in a
highly predictive way based on the observable features used in this
model.

\begin{verbatim}
Preparation of a new explainer is initiated
  -> model label       :  XGBoost Exit Model 
  -> data              :  24560  rows  30  cols 
  -> data              :  tibble converted into a data.frame 
  -> target variable   :  24560  values 
  -> predict function  :  yhat.workflow  will be used (  default  )
  -> predicted values  :  No value for predict function target column. (  default  )
  -> model_info        :  package tidymodels , ver. 1.3.0 , task classification (  default  ) 
  -> predicted values  :  numerical, min =  0.0001366138 , mean =  0.01243563 , max =  0.987971  
  -> residual function :  difference between y and yhat (  default  )
  -> residuals         :  numerical, min =  -0.3583511 , mean =  0.02396502 , max =  0.9993653  
  A new explainer has been created!  
\end{verbatim}

\begin{figure}

\centering{

\includegraphics{05-ML_final_files/figure-pdf/fig-custom-permutation-vi-1.pdf}

}

\caption{\label{fig-custom-permutation-vi}}

\end{figure}%

\section{Partial Dependence Profiles}\label{partial-dependence-profiles}

Figure (\textbf{ref?})(fig-pdp-top): Partial Dependence Profiles (PDPs)
for the six most influential predictors in the XGBoost model.

This figure presents Partial Dependence Profiles (PDPs), which visualize
the marginal effect of each individual feature on the predicted
probability of SKU exit, holding all other variables constant. The plots
reveal how changes in a single variable influence the model's average
prediction.

Key observations include:

click\_quantile\_overall: Exit probability decreases as visibility
increases. SKUs in higher click quantiles (i.e., with more historical
exposure) are less likely to exit.

max\_non\_interacting\_streak: The probability of exit rises sharply
with even short periods of inactivity. The risk plateaus after
approximately 10--15 days without interaction.

type\_2\_activated\_in\_product\_spacebrand\_ID: A larger number of new
third-party SKUs in the same brand-product space increases the
likelihood of exit, likely reflecting localized competition.

total\_sku\_in\_product\_space and
total\_sku\_type\_2\_in\_product\_space: Exit risk spikes in moderately
crowded product spaces (\textasciitilde200--300 SKUs), suggesting
crowding effects may peak at certain densities.

total\_sku\_type\_2\_in\_product\_spacebrand\_ID: Shows a U-shaped
pattern. Both low and high levels of competition within the
brand-product niche appear to increase exit probability, while moderate
levels pose less risk.

These profiles validate the model's emphasis on real-time engagement and
local market structure as key predictors of third-party SKU survival.

\begin{verbatim}
Preparation of a new explainer is initiated
  -> model label       :  XGBoost Exit Model 
  -> data              :  24560  rows  30  cols 
  -> data              :  tibble converted into a data.frame 
  -> target variable   :  24560  values 
  -> predict function  :  yhat.workflow  will be used (  default  )
  -> predicted values  :  No value for predict function target column. (  default  )
  -> model_info        :  package tidymodels , ver. 1.3.0 , task classification (  default  ) 
  -> predicted values  :  numerical, min =  0.0001366138 , mean =  0.01243563 , max =  0.987971  
  -> residual function :  difference between y and yhat (  default  )
  -> residuals         :  numerical, min =  -0.3583511 , mean =  0.02396502 , max =  0.9993653  
  A new explainer has been created!  
\end{verbatim}

\includegraphics{05-ML_final_files/figure-pdf/plot-dalex-pdp-top6-1.pdf}

\section{Conclusion}\label{conclusion}

This study set out to predict third-party (3P) SKU exit on JD.com
following first-party (1P) entry into product spaces, using a
combination of machine learning models and interpretability tools. While
the modeling pipeline was comprehensive---incorporating careful feature
engineering, multiple classification algorithms, and rigorous evaluation
through cross-validation---the predictive performance of all models was
limited, particularly in identifying rare exit events.

Our best-performing model, an XGBoost classifier trained with SMOTE and
class weighting, achieved moderate recall and F1-score but suffered from
high false positive rates. These outcomes suggest that third-party SKU
exit is not only rare but also difficult to anticipate based on
observable features alone. This unpredictability may reflect the
influence of unobserved strategic decisions, seller-specific
constraints, or platform-level interventions not captured in the
dataset.

As a result, the model's explanatory tools---such as feature importance
rankings and partial dependence plots---should be interpreted with
caution. While certain patterns (e.g., prolonged inactivity, high
crowding) appear consistently influential across multiple explanation
techniques, their practical relevance is limited by the model's low
precision and weak out-of-sample performance. In essence, when a model
cannot reliably predict an outcome, its explanations offer limited
actionable value.

Future research could benefit from richer longitudinal data,
seller-level covariates, and more precise identification of platform
entry events. Additionally, causal inference methods may complement
predictive models by isolating structural effects of platform
competition on seller survival. Until then, predictive modeling of 3P
exit on hybrid platforms remains a challenging task---highlighting the
complexity of strategic behavior in competitive online marketplaces.

\#Refrence

\phantomsection\label{refs}
\begin{CSLReferences}{1}{0}
\bibitem[\citeproctext]{ref-chi2022competition}
Chi, Y., Qing, P., Jin, Y. J., Yu, J., Dong, M. C., \& Huang, L. (2022).
Competition or spillover? Effects of platform-owner entry on provider
commitment. \emph{Journal of Business Research}, \emph{144}, 627--636.

\bibitem[\citeproctext]{ref-crawford2022amazon}
Crawford, G. S., Courthoud, M., Seibel, R., \& Zuzek, S. (2022).
\emph{Amazon entry on amazon marketplace}.

\bibitem[\citeproctext]{ref-deng2023can}
Deng, Y., Tang, C. S., Wang, W., \& Yoo, O. S. (2023). Can third-party
sellers benefit from a platform's entry to the market? \emph{Service
Science}, \emph{15}(4), 233--249.

\bibitem[\citeproctext]{ref-hu2024supercharged}
Hu, H., Qi, Y., Lee, H. L., Shen, Z.-J. M., Liu, C., Zhu, W., \& Kang,
N. (2024). Supercharged by advanced analytics, JD.com attains agility,
resilience, and shared value across its supply chain. \emph{INFORMS
Journal on Applied Analytics}, \emph{54}(1), 54--70.

\bibitem[\citeproctext]{ref-jiang2011firm}
Jiang, B., Jerath, K., \& Srinivasan, K. (2011). Firm strategies in the
{``mid tail''} of platform-based retailing. \emph{Marketing Science},
\emph{30}(5), 757--775.

\bibitem[\citeproctext]{ref-shen2024jd}
Shen, M., Tang, C. S., Wu, D., Yuan, R., \& Zhou, W. (2024). JD.com:
Transaction-level data for the 2020 MSOM data driven research challenge.
\emph{Manufacturing \& Service Operations Management}, \emph{26}(1),
2--10.

\bibitem[\citeproctext]{ref-song2020spillover}
Song, W., Chen, J., \& Li, W. (2020). Spillover effect of consumer
awareness on third parties' selling strategies and retailers' platform
openness. \emph{Information Systems Research}, \emph{32}(1), 172--193.

\bibitem[\citeproctext]{ref-zhu2018competing}
Zhu, F., \& Liu, Q. (2018). Competing with complementors: An empirical
look at amazon.com. \emph{Strategic Management Journal}, \emph{39}(10),
2618--2642.

\end{CSLReferences}




\end{document}
